\section{Introduction \label{sec:introduction}}

In the last few decades, ion-implanted \gls{bnmr} has established itself as unique microscopic probe of condensed matter~\cite{2015-MacFarlane-SSNMR-68-1}.
In many respects it is quite similar \gls{musr}~\cite{2004-Bakule-CP-45-203, 2011-Yaouanc-MSR},
with differences in the nuclear properties of the implanted probes making the two techniques complimentary rather than competitors~\cite{2000-Kiefl-PB-289-640}.

The most common \gls{bnmr} probe \ch{^{8}Li} often exhibits multi-exponential relaxation, complicating analysis.
Its high nuclear spin $I = 2$ is unique among (stable) \gls{nmr} nuclei, yielding fundamentally bi-exponential quadrupole relaxation~\cite{1970-Hubbard-JCP-53-985, 1982-Becker-ZNA-37-697, 1985-Korblein-JPFMP-15-561}.
In an experiment, especially in complex materials, it is not known beforehand if the multi-exponential polarization transients are due to:
multiple \ch{^{8}Li} stopping sites with distinct $T_{1}$s;
quadrupolar relaxation at a single site;
a distribution of relaxation times;
or combinations of the above possibilities.
In such cases, extreme care is required during the analysis when selecting a model to describe the data.
The same is especially true for the interpretation.
Both may be tackled through due diligence and exhaustive testing, but the process is hardly expedient.
It has therefore been desirable to have complementary means of addressing these uncertainties (\latin{e.g.}, what is the \emph{distribution} of $T_{1}$).
To this end, we explore the use of \glspl{ilt} as an alternative approach to analyzing \ch{^{8}Li} \gls{bnmr} \gls{slr} data.

Recall that \glspl{ilt} are useful when considering a signal that has the form:
\begin{equation}
\label{eq:signal-integral}
   y(t) = \int_{0}^{\infty} p(\lambda) \exp ( - \lambda t) \, \mathrm{d} \lambda,
\end{equation}
where $\lambda$ is the relaxation rate and $p ( \lambda )$ is its amplitude or probability density.
The implicit goal of the analysis is to determine $p ( \lambda )$, which can, in principle, be obtained by taking the \gls{ilt} of the transient $y(t)$:
\begin{equation}
\label{eq:ilt}
   p ( \lambda ) = \frac{1}{2 \pi \imath} \int_{c - \imath \infty}^{c + \imath \infty} y(t) \exp ( \lambda t ) \, \mathrm{d} t ,
\end{equation}
where $c$ is a real constant.

Evaluating the integral in \Cref{eq:ilt}, sometimes referred to as the Bromwich integral, is straightforward so long as $y(t)$ is known analytically;
however, \emph{this is almost never the case in experimental science}!
\textbf{See review on exponential analysis for more details~\cite{1999-Istratov-RSI-70-1233}.}
Pragmatically, $p ( \lambda )$ can only be found through solving \Cref{eq:signal-integral}.
Note that \Cref{eq:signal-integral} is a member of the more general class of Fredholm integral equations of the first kind.
These integrals are known to be ``ill'', ``incorrectly'', or ``improperly'' posed.
That is, the solution $p ( \lambda )$ obtained on solving \Cref{eq:signal-integral} may not be unique or even exist!

To deal with this \textbf{\dots more needed here!}
At this juncture, it is worth mentioning that using the term ``\gls{ilt}'' is somewhat inappropriate language, as discussed in detail elsewhere~\cite{2017-Fordham-DF-29-2}.
Nevertheless, we preserve the nomenclature for consistency with the \gls{nmr} literature.



The use of \glspl{ilt} is not uncommon in conventional \gls{nmr}, particularly when investigating environments of inherent complexity (\latin{e.g.}, in porous materials or petroleum deposits).


% Assuming only that our signal can be written as a sum of exponentially decaying terms, we.


The rest of the paper is organized as follows:
in \Cref{sec:theory} we outline the details of the ``\gls{ilt}'' analysis, including some particulars to \gls{bnmr} \gls{slr} data;
results from the approach are presented in \Cref{sec:results},
followed by a discussion in \Cref{sec:discussion}.
Finally, a concluding summary is given in \Cref{sec:conclusion}.
