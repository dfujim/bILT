\section{Results \label{sec:results}}

\begin{comment}
General plan of attack: 

Show how the ILT compares to simulated data, where the underlying function is absolutely well defined. We should compare with fits to the appropriate function.  

    We have a few relevant knobs to turn: 
        - Underlying functions: exponential, bi-exponential, tri-exponential, stretched exponential.
        - Spacing between the multi-exponential components. i.e. how far spread out are the different T1 values. What is the sensitivity? 
        - Relative amplitudes of the components. Do the peak heights actually reflect the proper weights? 
        - Amount of noise in the data. From the Monte-Carlo simulation, we can simply choose to run for a given number of counts.
        - Initial asymmetry is also a thing that affects things
       
    In all testing cases we may want to invent a new way of determining how well the transformation performs in producing the correct distribution. Because the 
    problem is ill-defined, I think that chisquared will always be pretty good. What we care about is therefore not the goodness of fit, but rather the goodness of 
    the resulting distribution. Here are some possibilities: 

        1. Fit the resulting distribution with a sum of gaussians (or something similar). Fix the mean value of the gaussian to the "True" T1 value of the underlying 
           distribution, then take a chisquared of this fit. Allowing the width of the gaussian to vary accounts for any issues with alpha being off by a bit. Hopefully 
           this tells how many peaks it finds and how close these peaks are to the true values. 

        2. Compute something like a standard deviation: sum_i [ (T1_true-T1_i)^2 * p_i ]. This may only work for single exp function though.  
        
        3. ???

Show how the ILT behaves with actual data

    Ideally I think this data should be published. We can then apply the ILT and compare to the result in the paper. We should aim for a really solid single exp, 
    probably a strexp, and maybe a biexp. 

    We could probably do the stats test with real data too. Just pick something that we know the functional form doesn't change, and see if there are large differences 
    in the resulting distributions. Even better if these are repeat measurements at the same settings.  

Figure list: 

    For one run: 
    1. Real SLR data + ILT fit function + normal fit function
    2. ILT probability distribution
    3. L curve
    4. The other diagnostic curves (or maybe just one, the most useful?)
    6. A plots showing the signature of: (a) strexp (b) const offset
    5. A couple of summary plots showing the goodness of fit (somehow, see above for options other than chisq) as a function of (a) function (b) spacing for biexp (c) noise level
    
    Probably topping out at 8 figures or something. It will be long perhaps. 
    

\end{comment}

