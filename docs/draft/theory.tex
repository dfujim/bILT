\section{Theory \label{sec:theory}}

In a \gls{bnmr} experiment, the experimental observable is the $\beta$-decay asymmetry $A(t)$, measured by the normalized difference in counting rates between two opposing detectors $N_{i}$.
Because of the parity violation in $\beta$-decay, $A(t)$ is proportional to nuclear spin-polarization $P(t)$ of the $\beta$-emitting probes.
That is,
\begin{equation}
   A(t) = A_{0} P(t),
\end{equation}
where the proportionality factor $A_{0}$ the depends on the details of the $\beta$-decay and the geometric details of the experiment.
The quantity of real interest is $P(t)$, whose temporal behaviour is connected with the fundamental properties of the host material.

In order to introduce the \gls{bnmr} probes in a host material of interest, they are ion-implanted using a \gls{dc} beam for a finite period of time $\Delta$, typically on the order of few seconds.
On arrival, each \gls{bnmr} probe interacts with electromagnetic fields inside the host and relaxes (\latin{i.e.}, re-orients);
however, not all probes arrive at the same time, which must be accounted for.
Fortunately, this convolution with the beam can be solved analytically with the time dependence of $P(t)$ written as~\cite{2006-Salman-PRL-96-147601, 2015-MacFarlane-PRB-92-064409}:
\begin{equation} \label{eq:polarization}
   P(t) = P_{0} \times
   \begin{cases}
      \frac{ \displaystyle \int_{0}^{t} \exp \left [ -\left ( t - t^{\prime} \right ) / \tau_{\beta} \right ] \, R \left (t,t^{\prime} \right ) \, \mathrm{d}t^{\prime} }{ \displaystyle \int_{0}^{t} \exp \left [ -t^{\prime} / \tau_{\beta} \right ] \, \mathrm{d}t^{\prime} }, & t \leq \Delta \\
      \frac{ \displaystyle \int_{0}^{\Delta} \exp \left [ - \left ( \Delta - t^{\prime} \right ) / \tau_{\beta} \right ] \,  R \left (t, t^{\prime} \right ) \, \mathrm{d}t^{\prime} }{ \displaystyle \int_{0}^{\Delta} \exp \left [ -t^{\prime} / \tau_{\beta} \right ] \, \mathrm{d} t^{\prime} }, & t > \Delta
   \end{cases}
\end{equation}
where $t^{\prime} < t$ is the arrival time of the probe, $P_{0}$ is the degree of polarization at $t = 0$ (determined by the optical pumping of \ch{^{8}Li} prior to implantation), $\Delta$ is the duration of the beam pulse, and $R \left (t, t^{\prime} \right )$ is the relaxation function.


Quite generally, $R \left (t, t^{\prime} \right )$ can be written as a distribution of decaying exponentials
\begin{equation} \label{eq:slr-integral}
   R \left (t, t^{\prime} \right ) = \int_{0}^{\infty} p ( \lambda ) \exp \left [-  \lambda \left ( t - t^{\prime} \right ) \right ] \, \mathrm{d} \lambda ,
\end{equation}
where $\lambda \equiv 1 / T_{1}$ is the \gls{slr} rate and $p ( \lambda )$ is the probability density.
This is quite analogous to \Cref{eq:signal-integral}.
In practice, spectra are often well-described (at the level of phenomenology) by a stretched exponential or \gls{kww} function~\cite{1854-Kohlrausch-AP-167-179, 1970-Williams-TFS-66-80, 1980-Lindsay-JCP-73-3348, 2006-Johnston-PRB-74-184430, 2016-Wu-SR-6-20506}.
In this case \Cref{eq:slr-integral} becomes:
\begin{equation} \label{eq:slr-stretched}
   R \left ( t, t^{\prime} \right ) \approx \exp \left \{ - \left  [ \lambda \left ( t-t^{\prime} \right ) \right ]^{\beta} \right \},
\end{equation}
where $\lambda \equiv 1/T_{1}$ is the relaxation rate and $0 < \beta \leq 1$ is the stretching exponent.
This form of $R \left ( t, t^{\prime} \right )$ is quite general and can arise from a continuous distribution of exponential relaxation times~\cite{2006-Johnston-PRB-74-184430}, with special cases of $\beta = 1/2$~\cite{1968-Tse-PRL-21-511, 1984-Stockmann-JNCS-66-501} or $\beta = 1/3$~\cite{1992-Bader-JPCM-4-4779} coming from \gls{slr} that is inhomogenously averaged in \gls{3d} or \gls{2d} limits.
The approach is often used in conventional solid-state \gls{nmr}~\cite{1995-Narayanan-JMRSA-112-58}, though chiefly as an expedient simplification of the multi-exponential magnetization transients, particularly when quadrupolar interactions are present~\cite{1995-McDowell-JMRSA-113-242}.
%In disordered or glassy materials like polymers a distribution of relaxation times is expected and a stretched exponential works works well~\cite{2014-McKenzie-JACS-136-7833, 2014-McGee-JPCS-551-012039, 2015-McKenzie-SM-11-1755, 2017-McKenzie-JCP-146-244903, 2018-McKenzie-SM-14-7324, 2019-Fujimoto-CM-31-9346}.
%While it may not be \latin{a priori} obvious why a relaxation transient appears stretched, the function has the following advantages as a phenomenological model:
%1) it accurately captures the characteristic $1/e$ relaxation time;
%and 2) it contains relatively few degrees of freedom (\latin{i.e.}, less than a biexponential model).


\textbf{A better transition needed here?}


We now begin the process of determining $p ( \lambda )$.
To start, we recast \Cref{eq:slr-integral} in discrete form, yielding
\textbf{(the notation should probably be improved\dots)}
%\begin{equation}
%\label{eq:signal-discrete}
%y ( t_{i} ) = \sum_{j} P( \lambda_{j} ) \exp ( - \lambda_{j} t_{i}) ,
%\end{equation}
%
\begin{equation} \label{eq:slr-sum}
   R \left (t_{i}, t^{\prime} \right ) = \sum_{j} p ( \lambda_{j} ) \exp \left [- \lambda_{j} \left ( t_{i} - t^{\prime} \right ) \right ] .
\end{equation}
%
With this transformation, we conveniently write $P(t)$ in matrix form:
%
\begin{equation} \label{eq:signal-matrix}
   \mathbf{P} = K \mathbf{p},
\end{equation}
%
where $K$ is the kernel matrix with elements
%
\begin{equation}
   \label{eq:kernel}
   [K]_{ij} \propto \exp \left ( - \lambda_{j} t_{i} \right )
\end{equation}
%
and $\mathbf{p}$ is the vector containing the associated amplitudes.
Of course, this can be easily re-cast as a vector of probability densities:
%
\begin{equation}
   \tilde{\mathbf{p}} = \frac{ \mathbf{p} }{ \sum_{i} [p]_{i} } .
\end{equation}
%


To solve the discretized problem, the goal is to
%
\begin{equation}
   \label{eq:onnls}
   \arg \min_{\mathbf{P} \geq 0} || \Sigma \left ( K \mathbf{p} - \mathbf{y} \right ) ||^{2} ,
\end{equation}
%
where $\Sigma$ is a diagonal matrix containing the uncertainies

This is a \gls{nnls} problem.
While straightforward, success is not guaranteed.
Matrix inversion is unstable \dots
\textbf{
Need to mention the details of the data itself:
histograms (\latin{i.e.}, discrete points in time) with inhomogeneous (statistical) uncertainties.
One has the freedom to adjust the time resolution of the spectra, up to a maximum determined at the time of data acquisition.
For \ch{^{8}Li}, a typical minimum bin width is \SI{10}{\milli\second}, which is a factor of \num{\sim 121} smaller than its nuclear lifetime.
}


To overcome this potentially show-stopping detail, \emph{regularization} is often introduced to increase the stability of the inversion.
This is done through the addition of a penalty term with a scale factor $\alpha$.
With this modification, \Cref{eq:onnls} becomes
%
\begin{equation}
\label{eq:rnnls}
   \arg \min_{\mathbf{p} \geq 0} || \Sigma(K \mathbf{p} - \mathbf{y}) ||^{2} + \alpha || \mathbf{p} ||^{2} .
\end{equation}
%
\Cref{eq:rnnls} is commonly known as Tikhonov regularization~\cite{1995-Tikhonov-NMSIPP} or ridge regression.
Effectively, the added term $\alpha || \mathbf{p} ||^{2}$ seeks to prevent the coefficients in $\mathbf{p}$ from adopting large values without incurring a penalty.
Pragmatically, this has the effect that, when $\alpha = 0$, $\mathbf{p}$ is most likely to contain elements close to 1 (\latin{i.e.}, giving a ``spiky'' solution).
Conversely, when $\alpha = \infty$, it forces the elements of $\mathbf{p}$ tend to zero, as $\alpha || \mathbf{p} ||^{2}$ is the dominant contribution to \Cref{eq:rnnls}.
As $\alpha$ grows, it tends to broaden $\mathbf{p}(\lambda)$.
Within the bounds $0 \leq \alpha \leq \infty$, there exists an ``optimum'' for the scale or smoothing parameter;
however, it is not known \latin{a priori} and must be determined for meaningful results.




Note, however, that \Cref{eq:rnnls} is \emph{not} the only form that the penalty ``add on'' can take!
Similarly, note that $\alpha$ \emph{must} be ``tuned'' in order for meaningful result to be obtained.


\Cref{eq:rnnls} can be solved using non negative least squares~\cite{1995-Lawson-SLSP}.



When it comes to finding the ``optimal'' $\alpha$, there are several approaches~\cite{2001-Kilmer-SIAMJMAA-22-1204}.
So, so many approaches~\cite{2011-Bauer-MCS-81-1795}, in fact!


One common approach is the so-called Butler-Reeds-Dawson method~\cite{1981-Butler-SIAMJNMA-18-381}, but this does not appear to apply here because our statistical errors/noise are intrinsically inhomogeneous across a spectrum.
\textbf{Is this the same as the ``the discrepancy principle''~\cite{1966-Morozon-DANSSR-167-510}?}
\textbf{It is possible this approach may work? See Eqs.~(7.9)--(7.10) on pp.~180--181 in~\cite{1998-Hansen-RDDIPP}.}
This is in contrast to, for example, an inversion-recovery NMR experiment, where the level of noise can be determined, in principle, by turning off the RF amplifier.



Independent of the minimization step, we can define the fit quality as:
\begin{equation}
   \chi^{2} = || \Sigma \left ( K \mathbf{p} - \mathbf{y} \right ) ||^{2} ,
\end{equation}
which is the familiar (weighted) sum of the squared residuals.




\gls{gcv} approach~\cite{1978-Craven-NM-31-377}:
Define
\begin{equation}
\label{eq:gcv}
   \min_{\alpha \geq 0} G ( \alpha ) = \frac{ || K \mathbf{p} - \mathbf{y} ||^{2} }{ \tau ( \alpha )^{2} } ,
\end{equation}
where
\begin{equation}
   \tau ( \alpha ) = \mathrm{Tr} \left [ \mathbf{I} - K \left ( K^{T} K + \alpha^{2} L^{T} L \right )^{-1} K^{T} \right ] .
\end{equation}
One can use an alternative method targeted at cases when $K$ is large~\cite{1997-Golub-JCGS-6-1}.
It is generally difficult to minimize $G ( \alpha )$ as it is generally flat.



Use L-curve analysis.
Visualize the size of the solution $|| \mathbf{p} ||$ vs.\ the residual $|| K \mathbf{p} - \mathbf{y} ||$ as a parametric plot for different $\alpha$.
Find the $\alpha$ that is the ``corner'' of the L-curve.
This is equivalent to to maximizing $\alpha$ up to the point where its becomes detrimental to the ``fit''.
In some cases where a minimum in $|| K \mathbf{p} - \mathbf{y} ||$ is pronounced, the $\alpha$ satisfying the ``corner'' condition is obvious.