%%%%%%%%%%%%%%%%%%%%%%%%%%%%%%%%%%%%%%%%%%%%%%%%%%%%%%%%%%%%%%%%%%%%%%%%%
\section{Miscellaneous \label{sec:miscellaneous}}

discussion of pitfalls~\cite{1983-Varah-SIAMJSSC-4-164}.

L-curve regularization~\cite{1992-Hansen-SIAMR-34-561, 1993-Hansen-SIAMJSC-14-1487}.
An earlier example~\cite{1990-Hansen-SIAMJSSC-11-503}?

Nice high-level article~\cite{1994-Craig-CP-8-648}.

bridge vs.\ lasso regularization methods~\cite{1998-Wenjiang-JCGS-7-397}.

\cite{2002-Venkataramanan-IEEETSP-50-1017}.

Review of \glspl{ilt} in noisy \gls{nmr} data~\cite{2013-Berman-CMRPA-42-72}.

Discussion of what exactly are \glspl{ilt}~\cite{2017-Fordham-DF-29-2}.

(see \latin{e.g.}, the classic textbook on the subject~\cite{1995-Lawson-SLSP}).

see also fairly recent multiexponential analysis review~\cite{1999-Istratov-RSI-70-1233} and numerical recipes~\cite{numerical-recipies}.

A recent, long review on regularization methods~\cite{2018-Benning-AN-27-1}.

Triangle method for finding the corner of the L-curve~\cite{2002-Castellanos-ANM-43-359}.

Example of using weights like we do~\cite{1999-Dunn-JMR-140-153}.

Some papers mention application to muons~\cite{1984-Honig-JCAM-10-113}.

People are still writing theses on regularization (see \latin{e.g.},~\cite{2011-OrozcoRodriguez-PhD}).

Paper with explicit re-casting of the equations being solved~\cite{2001-OLeary-SIAMJSC-23-1161} --- good to check against our implementation.

Another book by an authority on the topic~\cite{1998-Hansen-RDDIPP}.

%%%%%%%%%%%%%%%%%%%%%%%%%%%%%%%%%%%%%%%%%%%%%%%%%%%%%%%%%%%%%%%%%%%%%%%%%
\section{Implementation \label{sec:implementation}}

The \gls{nnls} algorithm solves equations of the form $L\mathbf{p} = \mathbf{z}$. The minimization \Cref{eq:rnnls} is accomplished using the following transformation of variables: 
%
\begin{align}
    L &= 
    \left(\begin{array}{c}
        \Sigma K \\ \Gamma
    \end{array}\right) &
    \mathbf{z} &= \left(\begin{array}{c}
        \Sigma\mathbf{y} \\ \bm{0}
    \end{array}\right),
\end{align}
%
where $K$ is defined by \Cref{eq:kernel}, $\Sigma$ is the diagonal matrix of reciprocal uncertainties, and $\Gamma$ is the regularization matrix. In general, the choice of $\Gamma$ is general, however in this work we define $\Gamma \equiv \alpha I$. It is trivial to then show that 
%
\begin{equation}
||\mathbf{z}-L\mathbf{p}||^2 = ||\Sigma (\mathbf{y} - K\mathbf{p})||^2 + ||\Gamma\mathbf{p}||^2.
\end{equation}
%
The \gls{nnls} optimization was performed via widely used Fortran subroutines~\cite{1995-Lawson-SLSP}, as implemented in Python, with the aid of the common scientific computing packages NumPy~\cite{2011-vanderWalt-CSE-13-22}, SciPy~\cite{2020-Virtanen-NM}, Matplotlib~\cite{2007-Hunter-CSE-9-90}, and Pandas~\cite{McKinney2010}.


%%%%%%%%%%%%%%%%%%%%%%%%%%%%%%%%%%%%%%%%%%%%%%%%%%%%%%%%%%%%%%%%%%%%%%%%%
\section{\glstext{bnmr} Data Simulation \label{sec:datasim}}

\Gls{mc} simulation was used to generate data sets where the underlying polarization function is fully known and absent of any distortion or artifacts due to technical considerations. Differences in detector geometry and placement, implantation rate, polarization effectiveness, and helicity all may be sources of distortion in the data. The probability that an electron is emitted at angle $\theta$ from the forward direction given by\cite{Correll1983}
%
\begin{equation}\label{eq:decay_direction}
W(t,\theta) = 1 + \frac{v}{c}AP(t)\cos(\theta)
\end{equation}
%
where $v$ is the velocity of the emitted electron, $c$ is the speed of light, $P(t)$ is the single particle polarization at time $t$ after implantation, and $A$ is the intrinsic beta-decay asymmetry unique to the nuclear species. For \ch{^8Li}, $A = -1/3$\cite{Arnold1988}. 

The implantation energies are uniformly distributed from $0$ to $\Delta$, as defined in \Cref{eq:polarization}. The decay times are exponentially distributed and the sum of the two is the beta detection time. The electron velocity arises from the distribution of decay energies~\cite{Mougeot2015}. Thus, \Cref{eq:decay_direction} can be used to calculate the decay direction, and if $3\pi/2<\theta<\pi/2$ then the particle is said to be detected by the forward detector, and otherwise it interacts with the backward detector. In this sense, the detectors ``work'' with perfect efficiency and coverage. The \gls{mc} simulation is implemented using C++ and  ROOT~\cite{Antcheva2009a}, and the \gls{pcg} random number generator.
