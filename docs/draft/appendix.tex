%%%%%%%%%%%%%%%%%%%%%%%%%%%%%%%%%%%%%%%%%%%%%%%%%%%%%%%%%%%%%%%%%%%%%%%%%
\section{Miscellaneous \label{sec:miscellaneous}}

discussion of pitfalls~\cite{1983-Varah-SIAMJSSC-4-164}.

L-curve regularization~\cite{1992-Hansen-SIAMR-34-561, 1993-Hansen-SIAMJSC-14-1487}.
An earlier example~\cite{1990-Hansen-SIAMJSSC-11-503}?

Nice high-level article~\cite{1994-Craig-CP-8-648}.

bridge vs.\ lasso regularization methods~\cite{1998-Wenjiang-JCGS-7-397}.

\cite{2002-Venkataramanan-IEEETSP-50-1017}.

Review of \glspl{ilt} in noisy \gls{nmr} data~\cite{2013-Berman-CMRPA-42-72}.

Discussion of what exactly are \glspl{ilt}~\cite{2017-Fordham-DF-29-2}.

(see \latin{e.g.}, the classic textbook on the subject~\cite{1995-Lawson-SLSP}).

see also fairly recent multiexponential analysis review~\cite{1999-Istratov-RSI-70-1233} and numerical recipes~\cite{numerical-recipies}.

A recent, long review on regularization methods~\cite{2018-Benning-AN-27-1}.

Triangle method for finding the corner of the L-curve~\cite{2002-Castellanos-ANM-43-359}.

Example of using weights like we do~\cite{1999-Dunn-JMR-140-153}.

Some papers mention application to muons~\cite{1984-Honig-JCAM-10-113}.

People are still writing theses on regularization (see \latin{e.g.},~\cite{2011-OrozcoRodriguez-PhD}).

Paper with explicit re-casting of the equations being solved~\cite{2001-OLeary-SIAMJSC-23-1161} --- good to check against our implementation.

Another book by an authority on the topic~\cite{1998-Hansen-RDDIPP}.

%%%%%%%%%%%%%%%%%%%%%%%%%%%%%%%%%%%%%%%%%%%%%%%%%%%%%%%%%%%%%%%%%%%%%%%%%
\section{Implementation \label{sec:implementation}}

We use:
NumPy~\cite{2011-vanderWalt-CSE-13-22},
SciPy~\cite{2020-Virtanen-NM},
and Matplotlib~\cite{2007-Hunter-CSE-9-90}.
The \gls{nnls} optimization was performed via widely used Fortran subroutines~\cite{1995-Lawson-SLSP}.

On implementation of \Cref{eq:rnnls} using the \gls{nnls} algorithm, one quickly realizes that the algorithm is useful only for solving equations of the form $L\mathbf{p} = \mathbf{z}$. This is resolved by defining the following variables in block form: 
%
\begin{align}
    L &= 
    \left(\begin{array}{c}
        \Sigma K \\ \Gamma
    \end{array}\right) &
    \mathbf{z} &= \left(\begin{array}{c}
        \Sigma\mathbf{y} \\ \bm{0}
    \end{array}\right),
\end{align}
%
where $K$ is defined by \Cref{eq:kernel}, $\Sigma$ is the diagonal matrix of reciprocal uncertainties, and $\Gamma$ is the regularization matrix. In general, the choice of $\Gamma$ is general, however in this work we define $\Gamma \equiv \alpha I$. It is trivial to then show that 
%
\begin{equation}
||\mathbf{z}-L\mathbf{p}||^2 = ||\Sigma (\mathbf{y} - K\mathbf{p})||^2 + ||\Gamma\mathbf{p}||^2.
\end{equation}
%

%%%%%%%%%%%%%%%%%%%%%%%%%%%%%%%%%%%%%%%%%%%%%%%%%%%%%%%%%%%%%%%%%%%%%%%%%
\section{\glstext{bnmr} Data Simulation \label{sec:datasim}}

\Gls{mc} simulation was used to generate data sets where the underlying polarization function is fully known, and which lack any distortion or artifacts due to technical considerations. Differences in detector geometry and placement, implantation rate, polarization effectiveness, and helicity all may be sources of distortion in the data. The probability that an electron is emitted at angle $\theta$ from the forward direction given by\cite{Correll1983}
%
\begin{equation}
W(t,\theta) = 1 + \frac{v}{c}AP(t)\cos(\theta)
\end{equation}
%
where $v$ is the velocity of the emitted electron, $c$ is the speed of light, $P(t)$ is the single particle polarization at time $t$ after implantation, and $A$ is the intrinsic beta-decay asymmetry unique to the nuclear species. For \ch{^8Li}, $A = -1/3$\cite{Arnold1988}. The cumulative probability distribution is given by 
%
\begin{equation}
W_\mathrm{CPD}(t,\theta) = \theta + \frac{v}{c}AP(t)\sin(\theta),
\end{equation}
%
which, for a given $t$, is invertible numerically. The time of implantation and decay are generated from the appropriate uniform and exponential distributions respectively, and their sum marks the beta detection time. If $3\pi/2<\theta<\pi/2$ then the particle is said to be detected by the forward detector, and otherwise it interacts with the backward detector. 